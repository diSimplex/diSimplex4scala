% This is a test of the DiSiTT parser targeted at 
% LaTeX proof style.

% My primary aim is to find a clarity of mathematical presentation. In the old
% styles, the objectives of proof checking and proof expostion were unified
% and/or mixed in together. With explicit computational proof checking (for all
% but the most-non-locally-finite steps (full Axiom of Choice?)), we do not
% really need the proof checking aspects of the classical proof.  We do however
% need a proof specification style which is easy to write in, but it is no
% longer \emph{required} to be expositional.
 
% Being able to split the proof specification from the proof expositional
% aspects of a given proof should allow for a simpler proof specification
% language and proof checking system.  It should also allow for a more fluid
% expositional style.

% \emph{However} the proof specification and proof expositional styles should
% not ``fight'' each other and become simply duplications of each other.

% The proof specification style will be a ``stack based'' language. This will
% suit both the bussproof.sty (Tree) and fitch.sty (List) based layouts of
% natural deductions. This will also make it easier to ``layout'' a proof
% specification in LaTeX.  In particular it should be easy to intermingle both
% the expositional ``text'' with the proof specifcational judgements so that
% LaTeX can format the two different styles in ``one'' readable format.  Note
% that it is highly likely that the author will refer implicitly to the
% judgements in their expositional text. This means that the resulting
% ``published'' document needs to keep the two contexts clear as well as
% inter-mingled.

% It is critical that we distinguish between explicit and implicit contexts.
% For ease of use, a human user would like to specify their proofs as implicitly
% as possible. The proof checker requires the proofs as explicitly as possible.
% To bridge this gap we need to make extensive use of implicit type checking to
% be able to expand the implicit specification into an explicit form.

% Since, in principle, every mathematical statement will be proven using this
% proof checker, we need tools to make it easier for a human user to search for
% proof specifications which might be useful in constructing their given proof
% at hand. This requires a comprehensive proof specification search engine
% and/or browser.

% A fundamental problem that needs to be addressed by this search engine, is
% that there is no canonical form for the proof judgements. Since we are
% addressing the full extent of mathematical proof, we can probabaly not expect
% to have a total language (such as Agda), and so we can probably not expect to
% have decidible normalization of the well-formed judgements. This will place a
% greater emphasis on a search engine that can interpret the semantical meaning
% of a judgement rather than \emph{just} the syntatic structure.

% Consider the following proof:

% \begin{theorem}
% The following conditions are equivalent:
% \begin{itemize}
%   \item $f$ is an isomorphism
%   \item $f$ is epic and split monic
%   \item $f$ is monic and split epic
% \end{itemize}
% \end{theorem}

% This proof is really a set of six possible proof specifications, one for each
% direction of implication between the three statements.  A typcial proof of
% this theorem would target three of these six possible proof specifications,
% leaving the remaining three proof specifications implicitly proven (via
% chaining the explicitly provided proof specifications). The fundamental
% problem with this ``style'' is that these proof specifications \emph{will} be
% required to form parts of subsequent proofs for a potentially wide range of
% users and purposes.  This means that the implicitly proven proof
% specifications are important to the over all proof community, even if they are
% ``obvious'' from the point of view of this theorem.

% There are two possible approaches to solve this ``problem''. We could
% implicilty form the remaining proof specifications by chaining the existing
% proof specifications and explicitly ``exporting'' these implicit proof
% specifications in the exposed theory package. Alternatively we could
% ``require'' the author to specify the complete collection of six proof
% specifications explicitly (or essentially the same thing not provide the
% implicit proof specifications to the greater proof community).  This explicit
% proof specification route is simpler to ``police'' and might lead to more
% performant proof checking. However the larger mathematical community would be
% better served by the implicit chaining approach in the longer run.  Such
% implicit chaining approach would have to look at performance issues.

% Mathematics produces a ``prodigious'' number of concepts about which it needs
% to discuss. The classical mathematical solution to this is to define ever more
% complex symbols to act as explicit and well defined ``words'' or ``names'' for
% these concepts. Once proven, a DiSiTT proof specification \emph{is} a new
% DiSiTT rule which can be used in subsequent natural deduction proofs
% specifications. This means that DiSiTT will need to be able to ``pattern''
% match any and all of these symbols structures. In particular this means that
% the DiSiTT grammar must be able to parse any and all symbols. Since LaTeX is
% able, by definition, to typeset any and all of these symbols, in the first
% instance the DiSiTT grammar will be able to parse LaTeX symbols syntax. The
% DiSiTT engine must then be able to pattern match on the resulting DiSiTT
% grammar AST.





