% \changes{0.1b}{2014/02/02}{Initial version}
%
% \iffalse
%<*never>
\documentclass{ltxdoc}
\usepackage{doc}
\usepackage{disitt-engine}
\usepackage{hyperref}
\newcommand{\define}[2]{\emph{\textbf{#1}}}
\GetFileInfo{disitt-engine.sty}
\CodelineIndex
\EnableCrossrefs
\RecordChanges
\begin{document}
\CheckSum{12}
\DocInput{disitt-engine.dtx}
\section{Copyright and \LaTeX\ Project Public License}
\begin{verbatim}
%txsBeginComment
Copyright (C) 2014 Stephen Gaito (PerceptiSys Ltd)

This work may be distributed and/or modified under the conditions of the
LaTeX Project Public License, either version  1.3  of  this license or (at
your option) any later version.

The latest version of this license is in
http://www.latex-project.org/lppl.txt and  version 1.3 or later is part of
all distributions of LaTeX version 2005/12/01 or later.

This work has the LPPL maintenance status `maintained'.

The Current Maintainer of this work is Stephen Gaito.

The released source can be found at:
https://github.com/diSimplex/diSimplexEngine/tree/master/texStyle  The

Development source can be found at:
https://github.com/stephengaito/diSimplexEngine/tree/master/texStyle

This work consists of the files disitt-engine.tex, 
disitt-engine-example.tex, disitt-engine.dtx and disitt-engine.ins 
and the derived files: disitt-engine.sty and disitt-engine.pdf.
%txsEndComment
% \end{verbatim}
\PrintIndex
\PrintChanges
\end{document}
%</never>
% \fi
%
% \DoNotIndex{\edef,\advance,\else,\csname,\endcsname,\expandafter}
% \DoNotIndex{\ifx,\fi,\input,\ifnum,\relax,\newcount,\romannumeral}
% \DoNotIndex{\the,\typeout,\xdef,\global,\newcommand,\RequirePackage}
% \DoNotIndex{\PackageError,\ProvidesPackage,\gdef}
% \DoNotIndex{\IncludeTests,\LogClose,\LogTests,\Expect,\ExpectIfThen}
% \DoNotIndex{\isnamedefined,\isundefined}
% \DoNotIndex{\stop,\makeatletter,\makeatother,\begin,\end,\def,\let}
% \DoNotIndex{\testFSDestroy@fifo@stack@count@top}
% \DoNotIndex{\testFSDestroy@fifo@stack@count@bottom}
% \DoNotIndex{\testFSDestroy@fifo@stack@count@size}
% \DoNotIndex{\fifo@stack@top@testFSDestroy}
% \DoNotIndex{\fifo@stack@bottom@testFSDestroy}
%
% \title{Directed Simplicial Type Theory \LaTeX\ Style}
% \author{Stephen Gaito}
% \maketitle
% \begin{abstract}
% This is the Directed Simplicial Type Theory (DiSiTT) style file. It is a LaTeX
% style file used to typeset DiSiTT proof arguments.
% \end{abstract}
% \tableofcontents
% \section{Introduction}
%
% The Directed Simplicial Type Theory (DiSiTT) style is a \LaTeX\ style designed
% to type set DiSiTT style natural decuction derivations, either rules or chains
% of rule derivations. The \emph{primary purpose} of using the DiSiTT style
% instead of other \LaTeX\ natural deduction packages is that the DiSiTT style
% provides the syntax required by the DiSiTT engine to perform automatic
% derivation checking. \emph{Inside} \LaTeX\ the DiSiTT style provides a
% unified syntax with in which either the ``Gentzen'' (tree) style (using
% \verb|bussproofs.sty|) or ``Fitch'' (sequential) style (using
% \verb|fitch.sty|) can be used.
%
% The Directed Simplicial Type Theory (DiSiTT) engine, is essentially a natural
% deduction judgement programming language (JPL). In classical natural deduction
% the basic \emph{judement} is a statement that a given proposition, $A$, is
% \emph{true}. In DiSiTT, the basic judgements are statements that one
% (directed) simplicial structure is a \emph{part} of another.
%
% Any programming language will need to keep track of variables in various
% contexts, our JPL is no different.  Natural deductions consist of nested
% collections of derivations or \emph{proofs}.  Variables are scoped to a
% particular nested collection of derivations.
%
% We have a number of distinct types of derivations:
%
% \begin{enumerate}
%   \item \define{Definition}{}\DescribeEnv{def}\ or
%     \define{Subdefinition}{}\DescribeEnv{subdef}\ used to \emph{define} an
%     equivilance between two statements. Subdefinitions are numbered relative to
%     their enclosing (sub)definition(s).
%   \item \define{Lemma}{} or \define{Theorem}{} used to provide a number of
%     \emph{hypothosized} statements followed by a number of \emph{derived}
%     statements. A Lemma or Theorem is a definition of a natural deduction rule
%     promise that a Proof will be given either in the following text, an
%     appendix, or from another work.
%   \item \define{Conjecture}{} the same as a Lemma or Theorem statement but
%     marked as unproven (and hence unuseable in other derivations).
%   \item \define{Proof}{} or \define{Subproof}{} used to provide detailed
%     natural deduction derivations of one or more \emph{derived} statements using
%     the given \emph{hypothosized} statements. Typically a proof follows a given
%     Lemma or Theorem.  However Proofs can be used as a running discusion on how
%     to prove a given statement (which might only be visible by the end of the
%     discussion).
% \end{enumerate}
%
% One of our primary aims is to produce ``journal'' ready documents, for this we
% use \LaTeX\ as our primary document environment and assume that the JPL
% derivations are environments inside a \LaTeX\ document. As such each of the
% above derivation types are defined as \LaTeX\ environments, for example:
%
% \begin{verbatim}
%   \begin{definition}
%     \begin{subdefinition}
%     ....
%     \end{subdefinition}
%   \end{definition}
% \end{verbatim}
%
% Each derivation environment consists of a number of derivation statements. A
% single statement consists of an optional \define{context}{} followed by a
% \define{judgement}{}, where a context is a sequence of judgements:
%
% \begin{verbatim}
%   \hypothesis{
%     \judgement{a}{A}, 
%     \judgement{b}{B},
%      ...
%     \judgement{x}{X}
%   }{
%     \judgement{z}{Z}
%   }
% \end{verbatim}
% or
% \begin{verbatim}
%   \conclusion{
%     \judgement{a}{A}, 
%     \judgement{b}{B},
%      ...
%     \judgement{x}{X}
%   }{
%     \judgement{z}{Z}
%   }
% \end{verbatim}
%
% Every derivation environment must have at least one \verb|\conclusion|
% statement. Multiple \verb|\conclusion| statements are considered to be
% mutulally
% equivilant. Derivation environments need not have any \verb|\hyposthesis|
% statements.  The order of statements is significant, and acts as a stack.
%
% In a proof derivation, every \verb|\conclusion| statement must be followed by
% a
% \verb|\followsUsing{xxx}| statement, which identifies the previously defined
% derivation which justifies the previous conclusion.
%
% \section{Examples}
%
% In this section we provide some worked examples.
%
% TODO: We eventually need to deal with the ``The following are equivilant''
% case with multiple conclusions.
%
% \begin{deAxiom}
%   \conclusion{}{\judgement{a}{A}}
% \end{deAxiom}
%
% \section{Integration tests and example usage}
%
% We use the \href{http://www.ctan.org/pkg/qstest}{QSTest} package from
% \href{http://www.ctan.org}{CTAN} to provide both integration and unit tests of
% our FIFO/stack package.
%
% In this section we walk through a number of integration tests which also
% provide example uses of FIFO/Stacks. We provide unit tests of specific
% invariants in the code section below.  These unit tests are associated
% with the section of code which implements a particular invariant.
%
% We begin by setting up the \LaTeX\ QSTest package to test the
% \verb|fifo-stack| package, and we will log everything. Note that we do not use
% a document
% class
% or begin/end a document, this is because there should not be any \emph{normal}
% output created.  All \emph{results} are listed in the associated
% \verb|fifo-stack-test.lgout| file.
%
%    \begin{macrocode}
%<*qstest>
\RequirePackage{qstest}
\RequirePackage{xifthen}
\IncludeTests{*}
\LogTests{lgout}{*}{*}
\RequirePackage{disitt-engine}
%</qstest>
%    \end{macrocode}
%
% \subsection{DiSittFS based examples}
%
% We now provide an example of standard use of the \verb|DiSittFS| FIFO/Stack
% based functionality. To do this we create a stack inside a stack and for each
% stack we push one thing onto the stack all in a nice linear fashion.
%
%    \begin{macrocode}
%<*qstest>
%\begin{qstest}{Simple DiSittFS FIFO/stack integration test}
%  {\CreateDiSittFS, \DestroyDiSittFS, 
%   \DiSittPush, \DiSittPop, \DiSittTop, \DiSittSize}
%  
%  \CreateDiSittFS{defaultValueA}
%  \Expect*{\DiSittTop}{defaultValueA}
%  \Expect*{\DiSittSize}{0}
%  \DiSittPush{firstValueFirstStack}
%  \Expect*{\DiSittTop}{firstValueFirstStack}
%  \Expect*{\DiSittSize}{1}
%  \CreateDiSittFS{defaultValueB}
%  \Expect*{\DiSittTop}{defaultValueB}
%  \Expect*{\DiSittSize}{0}
%  \DiSittPush{firstValueSecondStack}
%  \Expect*{\DiSittTop}{firstValueSecondStack}
%  \Expect*{\DiSittSize}{1}
%  \DestroyDiSittFS
%  \Expect*{\DiSittTop}{firstValueFirstStack}
%  \Expect*{\DiSittSize}{1}
%  \DiSittPop
%  \Expect*{\DiSittTop}{defaultValueA}
%  \Expect*{\DiSittSize}{0}
%\end{qstest}
%</qstest>
%    \end{macrocode}
%
% \subsection{A deAxiom example}
%
% We now provide an full deAxiom example.
%
%    \begin{macrocode}
%<*qstest>
%\begin{qstest}{First deAxiom integration test}
%  {\CreateDiSittFS, \DestroyDiSittFS, 
%   \DiSittPush, \DiSittPop, \DiSittTop, \DiSittSize}
%
%  \CreateDiSittFS{}
%  \conclusion{}{\judgement{a}{A}}
%  \Expect*{\DiSittSize}{1}
%  \Expect*{\DiSittTop}*{\ensuremath{ \vdash \judgement{a}{A}}}
%  \ifthenelse{1 < \DiSittSize}{\whiledo{1 < \DiSittSize}{\AxiomC{\DiSittShift}}}{\AxiomC{}}
%  \UnaryInfC{\DiSittShift}
%  \DisplayProof
%  \DestroyDiSittFS
%\end{qstest}
%</qstest>
%    \end{macrocode}
%
% \section{The \LaTeX\ code}
%
% \subsection{Preamble}
%
%    \begin{macrocode}
%<*package>
\ProvidesPackage{disitt-engine}%
  [2014/03/25 v1.0 DiSiTT natural deduction engine]
%txs#include:bussproofs
\RequirePackage{bussproofs}
%txs#include:ifthen
\RequirePackage{ifthen}
%txs#include:fifo-stack
\RequirePackage{fifo-stack}
%    \end{macrocode}
%
% \subsection{Meta-theory, theory and object levels}
%
% One of the fundamental distinctions for the DiSiTT engine is that between the
% various theoretical levels at which the engine is ``working''. There are three
% levels: %
% \begin{itemize} %
%    \item Meta-theory %
%    \item Theory %
%    \item Object %
% \end{itemize} %
% We use the \verb|\mTheory|\DescribeMacro{mTheory}{} and
% \verb|\oTheory|\DescribeMacro{oTheory}{} commands to denote objects at
% the \emph{meta-theory} and \emph{object-theory} levels respectively. Objects
% at the \emph{theory} level are unaddorned.
%
%    \begin{macrocode}
%txs\mTheory{object}#
\newcommand{\mTheory}[1]{\ensuremath{\overline{#1}}}
%txs\oTheory{object}#
\newcommand{\oTheory}[1]{\ensuremath{\underline{#1}}}
%    \end{macrocode}
%
% \subsection{The derivation environments}
%
% \subsubsection{Definition environments}
%
% \begin{environment}{deDefinition}
% The \define{deDefinition}{} environment
%    \begin{macrocode}
%txs\begin{deDefinition}#
%txs\end{deDefinition}#
\newenvironment{deDefinition}{begin}{end}
%    \end{macrocode}
% \end{environment}
%
% \begin{environment}{deSubdef}
% The \define{deSubdefinition}{} environment
%    \begin{macrocode}
%txs\begin{deSubdef}#
%txs\end{deSubdef}#
\newenvironment{deSubdef}{begin}{end}
%    \end{macrocode}
% \end{environment}
%
% \subsubsection{The Theorem environments}
%
% \begin{environment}{deAxiom}
% The \define{deAxiom}{} environment
%    \begin{macrocode}
%txs\begin{deAxiom}#
%txs\end{deAxiom}#
\newenvironment{deAxiom}{\StartNaturalDeduction}{\EndNaturalDeduction}
%    \end{macrocode}
% \end{environment}
%
% \begin{environment}{deLemma}
% The \define{deLemma}{} environment
%    \begin{macrocode}
%txs\begin{deLemma}#
%txs\end{deLemma}#
\newenvironment{deLemma}{%
begin-lemma%
\StartNaturalDeduction%
}{%
\EndNaturalDeduction%
end-lemma
}
%    \end{macrocode}
% \end{environment}
%
% \begin{environment}{deTheorem}
% The \define{deTheorem}{} environment
%    \begin{macrocode}
%txs\begin{deTheorem}#
%txs\end{deTheorem}#
\newenvironment{deTheorem}{begin}{end}
%    \end{macrocode}
% \end{environment}
%
% \subsubsection{The conjecture environment}
%
% \begin{environment}{deConjecture}
% The \define{deConjecture}{} environment
%    \begin{macrocode}
%txs\begin{deConjecture}#
%txs\end{deConjecture}#
\newenvironment{deConjecture}{begin}{end}
%    \end{macrocode}
% \end{environment}
%
% \subsubsection{The proof environments}
%
% \begin{environment}{deProof}
% The \define{deProof}{} environment
%    \begin{macrocode}
%txs\begin{deProof}#
%txs\end{deProof}#
\newenvironment{deProof}{begin}{end}
%    \end{macrocode}
% \end{environment}
%
% \begin{environment}{deSubproof}
% The \define{deSubproof}{} environment
%    \begin{macrocode}
%txs\begin{deSubproof}#
%txs\end{deSubproof}#
\newenvironment{deSubproof}{begin}{end}
%    \end{macrocode}
% \end{environment}
%
% \subsection{Judgement commands}
%
% \begin{macro}{\context}
%    \begin{macrocode}
%txs\context{someContext}#
\newcommand{\context}[1]{\ensuremath{#1}}
%    \end{macrocode}
% \end{macro}
%
% \begin{macro}{\judgement}
%    \begin{macrocode}
%txs\judgement{partOf}{container}#
\newcommand{\judgement}[2]{\ensuremath{#1\!\!:\!\!#2}}
%    \end{macrocode}
% \end{macro}
%
% \begin{macro}{\cJudgement}
%    \begin{macrocode}
%txs\cJudgement{contextList}#
\newcommand{\cJudgement}[1]{\ensuremath{( #1 ) \; \textbf{ctx}}}
%    \end{macrocode}
% \end{macro}
%
% \subsection{Hypothesis and Conclusion commands}
%
% \begin{macro}{\hypothesis}
%    \begin{macrocode}
%txs\hypothesis{context}{result}
\newcommand{\hypothesis}[2]{\pushHypothesis{\ensuremath{#1 \vdash #2}}}
%    \end{macrocode}
% \end{macro}
%
% \begin{macro}{\conclusion}
%    \begin{macrocode}
%txs\conclusion{context}{result}
\newcommand{\conclusion}[2]{\pushConclusion{\ensuremath{#1 \vdash #2}}}
%    \end{macrocode}
% \end{macro}
%
% \subsection{Support macros}
%
% For natural deduction derivations we have considered using one or more of:
% \begin{itemize}
%   \item \href{http://www.ctan.org/tex-archive/macros/latex2e/contrib/bussproofs}{BussProofs}
%   \item \href{http://www.mathstat.dal.ca/~selinger/fitch/}{Fitch}
%   \item \href{http://www.ctan.org/pkg/lplfitch}{LPLFitch}
%   \item \href{http://www.ctan.org/tex-archive/macros/latex/contrib/flagderiv/}{FlagDeriv}
% \end{itemize}
% The BussProofs and Fitch styles can be used without resorting to a FIFO or
% stack. Unfortunately, LPLFitch would require the use of a FIFO list
% architecture. Eqaully unfortunately, the Fitch style is not in CTAN or TeXLive
% (it can only be obtained from the author's website directly). Due to these
% considerations, we have, for the forseeable future chosen to use the
% \href{http://www.ctan.org/pkg/bussproofs}{BussProofs} packge from
% \href{http://www.ctan.org}{CTAN}.
%
% The semantics of the use of FlagDeriv is very clean. Unfortunately all of the
% Fitch-like packages put too much emphasis on the introduction/assumption
% ``step'' and not enough on the conclusion.
%
% While we \emph{have} chosen to use the BussProofs package, we explicitly wrap
% the BussProofs commands so that our LaTeXMarkDown parsing engine can easily
% identify the important structures. Unfortunately this means that we must
% auto-magically  keep track of the number of hypothesis in each DiSiTT
% derivation environment. We do this by maintaining a FIFO structure for each
% instantiated derivation environment. This means that we must count the number
% of hypothesis on a given FIFO structure.
%
% For lists and control sequence names we have considered the LaTeX
% \href{http://ctan.org/pkg/etoolbox}{etoolbox} package. However, we base our
% FIFO structure on ideas taken from Ben­jamin Ba­yart's excellent
% \href{http://www.ctan.org/tex-archive/macros/latex2e/contrib/stack}{LaTeX
% stack package}.  Unfortunately Bayart's stack package is explicitly only for
% stacks and we need a slightly more complex FIFO structure, hence we
% reimplement Bayart's stack package as an extended copy with the package name,
% ``\LaTeX FIFO/stack implementation for Package writers'' (fifo-stack),
% implemented for our specific needs.
%
% \begin{macro}{\StartNaturalDeduction}
%   \begin{macro}{\EndNaturalDeduction}
%    \begin{macrocode}
\makeatletter
\newcount\diSitt@saved@size
\@namedef{BussProofInf1}{\UnaryInfC}
\@namedef{BussProofInf2}{\UnaryInfC}
\@namedef{BussProofInf3}{\BinaryInfC}
\@namedef{BussProofInf4}{\TrinaryInfC}
\@namedef{BussProofInf5}{\QuaternaryInfC}
\@namedef{BussProofInf6}{\QuinaryInfC}
\newcommand{\StartNaturalDeduction}{\vspace{0.75ex}\CreateDiSittFS{}}
\newcommand{\EndNaturalDeduction}{%
  \diSitt@saved@size=\DiSittSize%
  \ifthenelse{1 < \DiSittSize}{%
    \whiledo{1 < \DiSittSize}{\AxiomC{\DiSittBottom\DiSittShift}}%
  }{%
    \AxiomC{}%
  }%
  \@nameuse{BussProofInf\the\diSitt@saved@size}{\DiSittBottom\DiSittShift}%
  \DisplayProof%
  \DestroyDiSittFS%
  \vspace{0.75ex}
}
\makeatother
%    \end{macrocode}
%   \end{macro}
% \end{macro}
%
% \subsubsection{FIFO structures}
%
% The FIFO/Stack package allows us to keep track of both the \emph{top} and
% \emph{bottom} of the  structure. We would \emph{like} our structures to be
% \emph{local} to a given \TeX\ group. Unfortunately, this might be broken by
% interviening \TeX\ groups, so the FIFO/Stack package retains Bayart's use of
% \emph{global} macros. However, we explicitly intend for DiSiTT derivation
% environments to be nested, so we must also keep track of how many nested FIFO
% structures are currently in use. To do this we implement, in this package, a
% FIFO/Stack of FIFO/Stacks.
%
% To begin our implementation of a Stack of FIFO/Stacks, we switch into `@'
% letter mode in order to facilitate our very low-level \TeX\ programming mode.
% We then create the global DiSiTT FIFO/Stack which will be used to keep track
% of the FIFO/Stacks associated with each instantiation of a DiSiTT derivation
% environment.
%    \begin{macrocode}
\makeatletter
\FSCreate{DiSittEnvironmentStack}{unknown}
%    \end{macrocode}
%
% Each (nested) derivation environment needs to call the
% \verb|\CreateDiSittFS|\DescribeMacro{CreateDiSittFS}\ and
% \verb|\DestroyDiSittFS|\DescribeMacro{DestroyDiSittFS}\ commands on
% \emph{begin} and \emph{end} respectively.
%
% \begin{macro}{\diSittStack@count}
%   \begin{macro}{\CreateDiSittFS}
%     \begin{macro}{\DestroyDiSittFS}
%    \begin{macrocode}
\newcount\diSittStack@count
\newcommand{\CreateDiSittFS}[1]{%
  \global\advance\diSittStack@count by 1 %
  \FSPush{DiSittEnvironmentStack}{diSittFS\the\diSittStack@count}
  \FSCreate{\FSTop{DiSittEnvironmentStack}}{#1}%
}
\newcommand{\DestroyDiSittFS}{%
  \FSDestroy{\FSTop{DiSittEnvironmentStack}}%
  \FSPop{DiSittEnvironmentStack}
}
%    \end{macrocode}
%     \end{macro}
%   \end{macro}
% \end{macro}
%
% Now that we have finished our very low-level \TeX\ programming, so we switch
% back to `@' other mode.
%    \begin{macrocode}
\makeatother
%    \end{macrocode}
%
% In order to simplify the use this Stack of FIFO/Stacks we provide wrapper FS
% operations which each operate on the top DiSiTT FIFO/Stack.
%
% \begin{macro}{\DiSittPush}
%   \begin{macro}{\DiSittPop}
%    \begin{macrocode}
\newcommand{\DiSittPush}[1]{\FSPush{\FSTop{DiSittEnvironmentStack}}{#1}}
\newcommand{\DiSittPop}{\FSPop{\FSTop{DiSittEnvironmentStack}}}
%    \end{macrocode}
%   \end{macro}
% \end{macro}
%
% \begin{macro}{\DiSittUnshift}
%   \begin{macro}{\DiSittShift}
%    \begin{macrocode}
\newcommand{\DiSittUnshift}[1]{\FSUnshift{\FSTop{DiSittEnvironmentStack}}{#1}}
\newcommand{\DiSittShift}{\FSShift{\FSTop{DiSittEnvironmentStack}}}
%    \end{macrocode}
%   \end{macro}
% \end{macro}
%
% \begin{macro}{\DiSittTop}
%   \begin{macro}{\DiSittBottom}
%    \begin{macrocode}
\newcommand{\DiSittTop}{\FSTop{\FSTop{DiSittEnvironmentStack}}}
\newcommand{\DiSittBottom}{\FSBottom{\FSTop{DiSittEnvironmentStack}}}
%    \end{macrocode}
%   \end{macro}
% \end{macro}
%
% \begin{macro}{\DiSittSize}
%    \begin{macrocode}
\newcommand{\DiSittSize}{\FSSize{\FSTop{DiSittEnvironmentStack}}}
%    \end{macrocode}
% \end{macro}
%
% \subsubsection{Judgement stack}
%
% \begin{macro}{\pushHypothesis}
%    \begin{macrocode}
%\newcommand{\pushHypothesis}[1]{\DiSittPush{#1}}
\newcommand{\pushHypothesis}[1]{{#1}}
%    \end{macrocode}
% \end{macro}
%
% \begin{macro}{\pushConclusion}
%    \begin{macrocode}
%\newcommand{\pushConclusion}[1]{\DiSittPush{#1}}
\newcommand{\pushConclusion}[1]{{#1}}
%    \end{macrocode}
% \end{macro}
%
% \subsection{Finishing off}
%
%    \begin{macrocode}
%</package>
%<*qstest>
\LogClose{lgout}
\stop
%</qstest>
%    \end{macrocode}
%
